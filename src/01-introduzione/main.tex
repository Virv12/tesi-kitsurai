\section{Introduzione}
\label{sec:introduzione}

%- motivazioni:
%  - importance of replicated storage services in Cloud Computing
%    ==> reliability
%  - increasing trend in using NoSQL storage solutions
%    => lower overheads & better scalability w.r.t. traditional SQL DBs
%       (much simpler operations)
%  - increasing importance of data access latency in emerging use-cases,
%    witnessed by DynamoDB ("single-digit millisecond latency"),
%    BigTable
%- providing guarantees in distributed, multi-tenant is difficult
%  => open-source? MongoDB, Cassandra -> best effort, no latency guarantees
%- emerging trend in adopting new languages like Go, Rust, ...
%  Rust interesting for safety, performance, ... ...???
%
%- first attempt at realizing a replicated key-value store with guaranteed
%  per-client/per-table performance, in the Rust programming language.

Negli ultimi anni, i servizi di storage replicato nel campo del Cloud Computing hanno acquisito molta importanza: garantire high-availability, fault-tolerance e consistenza dei dati è infatti fondamentale nel mercato moderno.
Sempre più frequentemente, le architetture moderne si affidano a soluzioni NoSQL, che, in cambio di operazioni più semplificate, offrono overhead inferiori e migliore scalabilità rispetto ai tradizionali database SQL.
Un altro aspetto sempre più imporatante è la latenza nell'accesso ai dati come osservato su piattaforme come DynamoDB che garantiscono tempi di risposta nell'ordine dei millisecondi.
Tuttavia, fornire garanzie di latenza e prestazioni in ambienti distribuiti e multi-tenant si rivela estremamente complesso. Le principali soluzioni open‑source come MongoDB o Cassandra adottano un approccio di tipo best effort, senza garantire limiti precisi sulla latenza o throughput, soprattutto quando più client condividono le risorse.
Il progetto Kitsurai si propone come primo tentativo di realizzare un key-value store distrubuito, in grado di offrire garanzie di throughput in un contesto multi-tenant. 
Il progetto è stato sviluppato in Rust per sfruttare le caratteristiche di sicurezza e prestazioni offerte dal linguaggio, e si propone di essere un'alternativa open-source alle soluzioni proprietarie come DynamoDB o BigTable.
