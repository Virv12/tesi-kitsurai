\section{Introduzione}
\label{sec:introduzione}

Negli ultimi anni, i servizi di storage replicato nel campo del \textit{Cloud Computing} hanno acquisito notevole importanza.
In un contesto in cui le applicazioni devono servire milioni di utenti contemporaneamente, spesso distribuiti geograficamente, diventa cruciale progettare sistemi in grado di garantire scalabilità orizzontale, alta disponibilità e tolleranza ai guasti.

Tali proprietà devono coesistere con l'esigenza di mantenere bassi tempi di risposta e garantire una gestione efficiente delle risorse.
I sistemi di gestione dati tradizionali, basati su modelli relazionali, risultano spesso inadatti a questi nuovi scenari distribuiti.
Per questo motivo, sempre più frequentemente, le architetture moderne si affidano a soluzioni \textit{NoSQL} che, in cambio di operazioni più semplificate, offrono un overhead inferiore e una migliore scalabilità rispetto ai tradizionali database \textit{SQL}.

Un altro aspetto di crescente importanza è la latenza nell'accesso ai dati, come osservato su piattaforme come \textit{DynamoDB} \cite{Dynamo2007}, che garantiscono tempi di risposta dell'ordine dei millisecondi \cite{DynamoWhitepaper}.
Tuttavia, fornire garanzie di latenza e prestazioni in ambienti distribuiti e \textit{multi-tenant} si rivela estremamente complesso.

Le principali soluzioni open-source, come \textit{MongoDB} \cite{Mongo} o \textit{Cassandra} \cite{Cassandra}, adottano un approccio di tipo \textit{best effort}, senza garantire limiti precisi sulla latenza o sul throughput, soprattutto quando più client condividono le risorse.
Il modello \textit{multi-tenant}, in cui più clienti condividono le stesse risorse fisiche o virtuali, introduce ulteriori sfide nella gestione del carico e nella garanzia di prestazioni minime.
In assenza di meccanismi espliciti di isolamento, un \textit{tenant} particolarmente attivo può compromettere la qualità del servizio offerto agli altri, rendendo essenziale un controllo preciso del throughput per ciascun cliente.

Il progetto \textit{Kitsurai} si propone come primo tentativo di realizzare un \textit{key-value store} che non solo sia scalabile e resiliente, ma che introduca anche meccanismi espliciti di controllo delle risorse, al fine di garantire un throughput minimo a ciascun tenant, anche in condizioni di carico elevato.

Il progetto è stato sviluppato in \textit{Rust}, linguaggio scelto non solo per le sue eccellenti prestazioni, ma anche per il modello di sicurezza della memoria che offre.
\textit{Rust} è noto per fornire performance comparabili a quelle del \textit{C++}, evitando però classi comuni di errori come \textit{race condition} o \textit{memory leak}, rendendolo particolarmente adatto alla programmazione di sistemi distribuiti complessi.

Nei capitoli che seguono verranno analizzate nel dettaglio le scelte progettuali, l'architettura del sistema e i risultati ottenuti attraverso test empirici, con particolare attenzione alle garanzie di throughput e latenza per i client.
