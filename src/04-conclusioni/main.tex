\section{Conclusioni}
\label{sec:conclusioni}

%Conclusions
%- works? really works?
%- limitations
%  => access control model
%  => table names?
%  => read conflicts? (lack of a catch-up protocol)

\subsection{Limitazioni}
\label{subsec:limitazioni}

Il progetto, pur dimostrando la validità dell'approccio proposto, presenta alcune limitazioni che richiedono ulteriori sviluppi. In particolare, l'implementazione attuale non prevede:

\begin{itemize}
    \item Un modello di controllo degli accessi per la gestione dei permessi.
    \item Un protocollo di catch-up per il reintegro dei nodi dopo un'interruzione.
    \item Meccanismi efficaci per la risoluzione dei conflitti in fase di lettura.
    \item L'indentificazione delle tabelle tramite nomi scelti dall'utente.
    \item Il supporto a macchine dotate di più dischi eterogenei.
    \item Strategie per la gestione ottimale della capacità dei nodi, soprattutto in scenari in cui le tabelle presentano requisiti sbilanciati in termini di banda e spazio di archiviazione.
    \item Il supporto per l'aggiunta e la rimozione dinamica dei nodi dal sistema.
\end{itemize}
