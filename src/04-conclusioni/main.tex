\section{Conclusioni}
\label{sec:conclusioni}

Il progetto \textit{Kitsurai} è stato sviluppato con l'obiettivo di realizzare un key-value store distribuito in grado di offrire prestazioni prevedibili e garantite in ambienti multi-tenant, un'esigenza spesso trascurata nelle principali soluzioni open-source esistenti.

I risultati sperimentali confermano la validità dell'approccio adottato: il sistema riesce a mantenere una latenza controllata anche in presenza di carichi elevati e client malevoli, garantendo un efficace isolamento tra i tenant e assicurando a ciascuno una quota minima di banda.

\subsection{Limitazioni}
\label{subsec:limitazioni}

Nonostante i risultati promettenti, l'implementazione attuale presenta alcune limitazioni che offrono spunti per futuri sviluppi. In particolare, il sistema non include ancora:

\begin{itemize}
    \item Un modello di controllo degli accessi per la gestione granulare dei permessi.
    \item Un protocollo di catch-up per il reintegro dei nodi dopo interruzioni prolungate.
    \item Meccanismi robusti per la risoluzione dei conflitti durante le operazioni di lettura.
    \item La possibilità di assegnare nomi personalizzati alle tabelle da parte degli utenti.
    \item Strategie per una gestione ottimale della capacità dei nodi, in particolare quando le tabelle hanno requisiti sbilanciati in termini di banda o spazio di archiviazione.
    \item Il supporto per l'aggiunta e la rimozione dinamica dei nodi dal sistema.
    \item Un protocollo RPC migliorato, basato su tecnologie già esistenti, ad esempio QUIC~\cite{QUIC}.
\end{itemize}
