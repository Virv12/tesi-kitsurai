\section{Conclusioni}
\label{sec:conclusioni}

%Conclusions
%- works? really works?
%- limitations
%  => access control model
%  => table names?
%  => read conflicts? (lack of a catch-up protocol)

Il progetto \textit{Kitsurai} nasce con l'obiettivo di realizzare un key-value store distribuito capace di offrire prestazioni prevedibili e garantite in ambienti multi-tenant, un aspetto spesso trascurato nelle principali soluzioni open-source. Il sistema proposto introduce un modello innovativo di controllo esplicito del throughput, abbinato a un'architettura scalabile e resiliente.

I risultati sperimentali ottenuti dimostrano la validità dell'approccio proposto. Il sistema è in grado di mantenere una latenza controllata anche sotto carichi elevati, riuscendo a isolare efficacemente i tenant e a garantire una quota di banda minima a ciascuno. Le politiche di allocazione e limitazione del throughput, così come l'uso consapevole delle availability zones, contribuiscono ulteriormente alla stabilità del sistema.

\subsection{Limitazioni}
\label{subsec:limitazioni}

Il progetto presenta alcune limitazioni che richiedono ulteriori sviluppi.
In particolare, l'implementazione attuale non prevede:

\begin{itemize}
    \item Un modello di controllo degli accessi per la gestione dei permessi.
    \item Un protocollo di catch-up per il reintegro dei nodi dopo un'interruzione.
    \item Meccanismi efficaci per la risoluzione dei conflitti in fase di lettura.
    \item L'indentificazione delle tabelle tramite nomi scelti dall'utente.
    \item Il supporto a macchine dotate di più dischi eterogenei.
    \item Strategie per la gestione ottimale della capacità dei nodi, soprattutto in scenari in cui le tabelle presentano requisiti sbilanciati in termini di banda e spazio di archiviazione.
    \item Il supporto per l'aggiunta e la rimozione dinamica dei nodi dal sistema.
    \item Milgioramenti al protocollo di RPC, magari usando QUIC\cite{QUIC}.
\end{itemize}
